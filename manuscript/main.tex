\documentclass[11pt]{article}
\usepackage{amsmath, amssymb}
\usepackage{geometry}
\usepackage{setspace}
\usepackage{hyperref}
\usepackage{graphicx}
\geometry{margin=1in}
\onehalfspacing

\title{The Sigma-Distribution Hypothesis: A Probabilistic Framework for Lifestyle Disease Onset, Aging, and Prevention}
\author{Anurag Garg \\ Independent Researcher}
\date{}

\begin{document}

% Title page information
\begin{titlepage}
\centering
\vspace*{1cm}
{\LARGE \textbf{The Sigma-Distribution Hypothesis: A Probabilistic Framework for Lifestyle Disease Onset, Aging, and Prevention}}
\vspace{1.5cm}

{\large \textbf{Anurag Garg}}
\vspace{0.5cm}

\textit{Independent Researcher}
\vfill
\vspace{2cm}

\textbf{Corresponding author:} \\
Anurag Garg \\
Email: anurag.garg@moe.sch.ae \\
\vspace{1cm}

\textbf{Keywords:} sigma distribution, preventive medicine, aging, lifestyle diseases, risk communication, survival analysis
\vspace{1cm}

\textbf{Word count:} ~3,100 words
\vfill
\end{titlepage}

\begin{abstract}
Lifestyle-related chronic diseases (e.g., hypertension, type 2 diabetes, and cardiovascular disease) emerge stochastically along age-dependent trajectories, yet prevention is typically framed through binary thresholds and static risk scores. We propose the \emph{sigma-distribution hypothesis}: within clinically relevant age windows, disease onset times can be treated as a population distribution, and an individual's evolving position in that distribution can be expressed in standard deviation (\(\sigma\)) units. Remaining disease-free while maintaining favorable and stable biomarkers updates the posterior onset distribution, shifting probability mass away from early onset and reducing near-term hazard. Reduced biomarker variability is expected to compress the effective left tail of risk and amplify this rightward shift. We illustrate computability with biomarker-based examples (including HbA1c for diabetes and systolic blood pressure for hypertension), using public epidemiologic reference parameters. The framework yields testable predictions and provides an interpretable metric for communicating preventive progress as movement toward later expected onset.
\end{abstract}

\section{Introduction}

Lifestyle-related chronic diseases--most notably hypertension, type 2 diabetes, and cardiovascular disease--emerge over long time horizons with substantial inter-individual variability in age of onset \cite{rothwell2010, gregg2014}. Preventive practice typically relies on binary thresholds or static absolute risk estimates, which provide limited insight into how disease-free survival and biomarker stability modify future risk as individuals age.

Epidemiologically, disease onset is described by age-dependent hazard functions \(\lambda(t)\), with risk conditioned on survival history \cite{cox1972, pencina2009}. Aging research emphasizes increasing heterogeneity of physiological systems over time, with stable regulation associated with delayed morbidity \cite{levine2018}. Yet these insights lack a unified, interpretable framework quantifying preventive progress in probabilistically rigorous yet clinically intuitive terms.

Current approaches focus on absolute values rather than relative position within population distributions. Two individuals with identical biomarker levels may occupy different probabilistic states depending on age and prior disease-free survival. Prevention is thus conceptualized as maintaining biomarkers below fixed thresholds rather than as movement within broader onset distributions.

We propose the \textbf{sigma-distribution hypothesis}, reframing prevention as a dynamic, distributional process. We posit that within clinically relevant age windows, disease onset times can be treated as population distributions, with individual positions expressed in standard deviation (\(\sigma\)) units. Remaining disease-free while maintaining favorable biomarkers updates the conditional onset distribution, shifting probability mass away from early onset. This rightward movement provides quantitative interpretation of preventive success consistent with survival analysis, variability research, and compression of morbidity \cite{fries1980}.

\section{The Hypothesis/Theory}

\subsection{Core Proposition}

The \textbf{sigma-distribution hypothesis} consists of three linked propositions.

\textbf{Proposition 1 (Local onset-time distribution):} For lifestyle diseases with age-dependent onset, the population distribution of onset ages can be approximated locally within clinically relevant age windows (e.g., 40-70 years) by a parametric or nonparametric form. \textbf{For interpretability and illustration}, a Gaussian form \(T \sim \mathcal{N}(\mu,\sigma^2)\) may be used, but real-world distributions are often right-skewed (e.g., Weibull, Gompertz). The sigma-positioning framework is \textbf{distribution-agnostic} and can be adapted using percentile-based z-scores or transformations.

\textbf{Proposition 2 (Dynamic positioning):} An individual's evolving position within the onset-time distribution can be expressed in standard deviation units and updated as time advances without diagnosis. Remaining disease-free provides conditional information (\(T>a\)) that shifts posterior probability mass away from early onset.

\textbf{Proposition 3 (Role of biomarker stability):} Beyond mean biomarker levels, reduced biomarker variability tightens an individual's effective risk profile, attenuating left-tail probability mass and amplifying rightward movement in sigma position.

\subsection{Mathematical Formalization}

Let $T$ be the random variable representing age of disease onset for a specific condition in a reference population, with estimated parameters $\hat{\mu}$ (mean onset age) and $\hat{\sigma}$ (standard deviation). For an individual at current age $a$:

\[
z(a) = z_0(a) + \Delta z_{\text{cumulative}}, \qquad z_0(a)=\frac{a-\hat{\mu}}{\hat{\sigma}}
\]
where $\Delta z_{\text{cumulative}}$ accumulates from biomarker stability and risk factors.

The conditional hazard at age $a$ becomes:
\[
\lambda(a \mid z) = \lambda_0(a) \cdot \exp(-\beta z)
\]
with $\beta > 0$ representing the protective effect of rightward sigma positioning. This proportional form is used as a conceptual mapping; no specific functional form is assumed.

\section{Evaluation of the hypothesis/idea}

\subsection{Empirical Foundations}

\subsubsection{Conditional survival inherently updates near-term risk}
Survival to age $a$ truncates early-onset probability mass, changing the posterior distribution of onset times. Cox-type models operationalize this through age-dependent hazards conditioned on survival history \cite{cox1972}.

\subsubsection{Biomarker variability adds information beyond mean levels}
Visit-to-visit blood pressure variability predicts cardiovascular events beyond mean blood pressure \cite{rothwell2010}, and glycemic variability associates with oxidative stress and microvascular complications beyond HbA1c alone \cite{monnier2006}. Similar principles apply to lipid variability, inflammatory markers, and other risk indicators across multiple chronic diseases. These findings motivate Proposition 3: reduced variability indicates tighter risk profiles.

\subsubsection{Delayed onset and compression of morbidity}
Sigma-distribution framing provides quantitative interpretation of delayed onset compatible with compression-of-morbidity \cite{fries1980, fries2011}. Successful prevention shifts probability mass to later ages, compressing morbidity burden across multiple conditions including diabetes, hypertension, and cardiovascular disease.

\subsubsection{Early-onset phenotypes and posterior reclassification}
Earlier onset often reflects more aggressive phenotypes across multiple diseases \cite{hillier2001}. As individuals remain disease-free, posterior probability of belonging to early-onset subgroup declines. Sigma-position makes this reclassification interpretable for various conditions.

\subsection{Theoretical Consistency}

\subsubsection{Local distributional approximations are standard}
Local parametric approximations over restricted age windows are widely used for interpretability. Gaussian form is a pragmatic approximation, not a claim of global normality.

\begin{figure}[htbp]
\centering
\includegraphics[width=0.9\textwidth]{figure1_sigma_hba1c.png}
\caption{\textbf{Illustrative sigma positioning as an age-conditioned coordinate system.}
(A) Schematic age-specific reference distribution for a biomarker, showing an age-dependent mean (solid line) and $\pm 1\sigma$ and $\pm 2\sigma$ bands (shaded). Points illustrate that the same absolute biomarker value can correspond to different sigma positions at different ages. A percentile trajectory is shown for conceptual contrast between absolute and relative positioning. (B) Corresponding sigma trajectories under the two conceptual scenarios. \textbf{This figure is for conceptual and computational illustration only}; it does not model disease incidence, does not estimate onset-time distributions from longitudinal cohorts, and should not be interpreted as outcome prediction or clinical validation.}
\label{fig:sigma_illustration}
\end{figure}

\textbf{Figure \ref{fig:sigma_illustration}} provides a visual demonstration of sigma positioning, showing how identical absolute values yield different z-scores at different ages, and contrasting absolute versus relative positioning trajectories.

\textbf{Note on distributional form}: While we use a Gaussian form for visual and conceptual simplicity, onset-age distributions in chronic diseases are typically right-skewed. In applied settings, sigma positioning should use \textbf{empirical percentiles} or transformed z-scores from fitted survival distributions (e.g., Weibull, log-normal) to avoid bias. The core idea—relative positioning within a reference distribution—remains valid irrespective of distribution shape.

\subsection{Computational demonstration (illustrative, non-clinical)}
\label{sec:computational_demo}

To reduce ambiguity about scope, we include a computational demonstration intended solely to show that sigma-positioning is (i) computable from age-stratified reference distributions, (ii) numerically stable, and (iii) able to generate interpretable visualizations. This is \textbf{not} a clinical validation study, does not estimate disease incidence, and does not claim predictive or diagnostic utility.

Two demonstration components are used. First, a \textbf{synthetic-data check} verifies internal coherence under a known generative model: when a biomarker’s age-dependent mean and variance are embedded into the data-generating process, re-parameterizing from an absolute scale to an age-conditioned sigma coordinate can change statistical discrimination in a controlled setting. Second, a \textbf{cross-sectional population illustration} uses a public epidemiologic dataset to visualize (a) how biomarker distributions drift with age and (b) how sigma positions behave as a well-defined coordinate system. Because the population illustration is cross-sectional and uses prevalence labels, it should be interpreted only as feasibility of the transformation, not as an onset-time model.

All code for regenerating the illustrative figures, along with versioned dependencies and output artifacts, is provided in the accompanying repository (see Section~\ref{sec:implementation}).


\subsubsection{Physiological stability as a systems-level signal}
From complex-systems perspective, stable regulation reflects resilient control dynamics across multiple physiological domains \cite{seely2004}. Sigma-distribution hypothesis translates this stability signal into distributional language operationalizable with longitudinal data for various biomarkers and conditions.

\subsubsection{Compatibility with multidimensional aging measures}
Biological age estimators implicitly place individuals within multivariate reference distributions \cite{levine2018}. Sigma-distribution hypothesis is compatible but emphasizes interpretable, outcome-adjacent target: delaying onset distributions across multiple age-related diseases.

\begin{table}[htbp]
\centering
\caption{Illustrative Biomarker Sigma Positioning at Different Ages}
\begin{tabular}{c c c c c c}
\hline
Biomarker & Age (years) & Value & $\mu(a)$ & $\sigma(a)$ & $z$ \\
\hline
HbA1c (\%) & 40 & 5.6 & 5.4 & 0.3 & +0.67 \\
HbA1c (\%) & 60 & 5.6 & 5.7 & 0.4 & -0.25 \\
SBP (mmHg) & 50 & 125 & 128 & 10 & -0.30 \\
SBP (mmHg) & 70 & 125 & 135 & 12 & -0.83 \\
LDL (mg/dL) & 45 & 110 & 115 & 20 & -0.25 \\
LDL (mg/dL) & 65 & 110 & 125 & 22 & -0.68 \\
\hline
\end{tabular}
\label{tab:biomarker_examples}
\end{table}

\textbf{Table \ref{tab:biomarker_examples}} illustrates how identical absolute biomarker values correspond to different sigma positions at different ages across multiple risk indicators (HbA1c for diabetes, systolic blood pressure [SBP] for hypertension, LDL cholesterol for cardiovascular disease), highlighting the age-conditioned interpretation central to the sigma-distribution hypothesis.

\subsection{Practical implementation steps}
\label{sec:implementation}

Implementation of the sigma-distribution framework in practice involves the following steps:

\begin{enumerate}
    \item \textbf{Reference distribution estimation:}
    Use interval-censored survival models (e.g., Weibull, Cox proportional hazards with baseline hazard estimation) from cohort data to estimate the conditional distribution of disease onset ages given current age and biomarker levels. For interpretability, these can be summarized as location-scale distributions or as nonparametric percentiles.

    \item \textbf{Sigma calculation:}
    For an individual at age \(a\) with biomarker value \(x\), compute the sigma position as:
    \[
    z(a) = \Phi^{-1}\big( F_{T \mid a}(x) \big),
    \]
    where \(F_{T \mid a}\) is the estimated cumulative distribution function of onset age conditional on reaching age \(a\) without disease, and \(\Phi\) is the standard normal cumulative distribution function (if using z-scores). If the underlying distribution is non-Gaussian, percentile-based positioning can be used directly.

    \item \textbf{Longitudinal updating:}
    Update \(z(a)\) iteratively as new biomarker measurements and survival information (remaining disease-free) arrive. This can be implemented using Bayesian updating of the posterior onset distribution or through repeated fitting of conditional survival models.

    \item \textbf{Software implementation:}
    Example code for estimating onset-age distributions, calculating sigma positions, and visualizing trajectories for multiple biomarkers is provided in the supplementary repository: \url{https://github.com/mranuraggarg/sigma-distribution-hypothesis}.
\end{enumerate}

\subsection{Relation to conventional risk scores}
\label{sec:risk-score-comparison}

Existing clinical risk scores (e.g., Framingham, QRISK3, ACC/AHA Pooled Cohort Equations) output absolute probabilities or risk categories over fixed time horizons for specific diseases. The sigma-distribution framework complements these approaches by:

\begin{itemize}
    \item Providing a \textbf{relative, distributional position} within an age-matched reference across multiple biomarkers, rather than disease-specific absolute risk estimates.
    \item \textbf{Explicitly incorporating longitudinal biomarker stability} as a modifier of risk position applicable to various conditions.
    \item Enabling \textbf{dynamic updating} with each additional disease-free year, reflecting survival conditioning in real time for multiple potential outcomes.
\end{itemize}

Sigma positioning could be integrated with existing scores by converting their outputs into distributional percentiles, offering a more granular and updatable view of risk evolution across the spectrum of lifestyle diseases. Future empirical work should compare the predictive performance of sigma-based metrics against established risk scores for near-term onset of diabetes, hypertension, cardiovascular disease, and other age-related conditions.

\section{Hypothesis testing}

The hypothesis can be evaluated using longitudinal cohort data and prospective studies. Because the present manuscript is a hypothesis paper, we outline an evaluation strategy rather than reporting validated performance. The primary test is whether sigma-position (and sigma drift over time) provides incremental information beyond conventional risk factor summaries when predicting \textbf{subsequent} incident disease in longitudinal settings.

\subsection{Cross-sectional validation}
Compute sigma positions from age-conditioned biomarker distributions in disease-free subpopulations. Test whether sigma position at fixed age predicts subsequent incidence of diabetes, hypertension, cardiovascular disease, and other conditions beyond conventional risk factors.

\subsection{Variability-sigma relationship}
Quantify within-person biomarker variability for multiple risk indicators (blood pressure, glucose, lipids) and relate it to sigma drift over time. Hypothesis predicts variability measures correlate with sigma position changes across different disease domains, adding explanatory power beyond mean levels.

\subsection{Intervention studies}
For interventions with similar effects on mean biomarker levels across different conditions, compare those reducing variability versus those not. Hypothesis predicts variability-reducing interventions produce larger rightward sigma shifts for multiple disease outcomes.

\subsection{Concordance with aging clocks}
Test correlation between sigma position for different biomarkers and established biological age estimators (e.g., PhenoAge, DunedinPACE) independently of chronological age across diverse populations.

\subsection{Longitudinal trajectory analysis}
Track sigma positions for multiple biomarkers over decades in longitudinal cohorts. Hypothesis predicts individuals maintaining higher sigma positions across different risk indicators experience delayed onset and compressed morbidity for various age-related diseases.

\section{Empirical data}

No new empirical data are presented. The illustrative examples use synthetic parameters consistent with population characteristics to demonstrate computational feasibility across multiple biomarkers and conditions. This approach is appropriate for a hypothesis paper and avoids claims of clinical validation.

\section{Consequences of the hypothesis and discussion}

If supported, the sigma-distribution hypothesis would:

\subsection{Provide alternative endpoints for prevention studies}
Beyond hazard ratios, trials could report mean changes in sigma position as interpretable, standardized endpoint capturing both level and stability effects across multiple biomarkers and disease outcomes.

\subsection{Enable personalized feedback and trajectory tracking}
Individuals and clinicians could monitor sigma position longitudinally across different risk indicators, communicating preventive progress as movement within age-matched distributions for various conditions.

\subsection{Improve risk communication}
Statements like "movement from median toward higher percentiles of onset distribution" may be more intuitive than isolated absolute risk percentages for communicating risk across multiple diseases.

\subsection{Guide precision prevention}
Framework separates objectives: shifting mean biomarker levels versus stabilizing variability across different risk indicators, guiding intervention selection when mean values are acceptable but variability remains high for specific conditions.

\subsection{Limitations and potential failure modes}
\begin{enumerate}
\item \textbf{Distributional misspecification:} Onset ages may be skewed or multimodal across different diseases; local approximations should be checked empirically for each condition.
\item \textbf{Population specificity:} Reference parameters vary by cohort, era, and context; sigma values are meaningful only relative to defined reference populations for specific diseases.
\item \textbf{Measurement requirements:} Quantifying within-person variability requires repeated measurements across multiple biomarkers and careful error handling.
\item \textbf{Competing risks:} Mortality and differential screening can censor observed onset times across different disease outcomes.
\item \textbf{Equity and representativeness:} Reference distributions must be population-specific across different demographic groups. Using non-representative references could exacerbate health disparities by misclassifying risk in underrepresented groups for various conditions.
\item \textbf{Dynamic references:} Onset distributions may shift over time due to public health trends, treatments, or environmental changes across different diseases, requiring periodic updates to reference data.
\item \textbf{Disease interactions:} The framework currently treats diseases independently; in practice, multiple conditions may interact, complicating sigma position interpretation.
\end{enumerate}

\subsection{Clinical translation and communication barriers}
\label{sec:translation-barriers}

While the sigma-distribution framework offers conceptual advances applicable to multiple chronic diseases, several practical barriers must be addressed for clinical adoption:

\begin{itemize}
    \item \textbf{Interpretability:} Sigma units and distributional thinking may be unfamiliar to patients and clinicians across different disease contexts. Visual aids (e.g., percentile sliders, risk curves, age‑conditional density plots) will be essential for effective communication across various conditions.
    \item \textbf{Electronic health record (EHR) integration:} Real‑time calculation of sigma positions for multiple biomarkers requires embedding reference distributions and computational routines within EHR systems, posing technical and interoperability challenges.
    \item \textbf{Training and guideline adaptation:} Clinicians will need guidance on interpreting sigma shifts as markers of preventive progress across different disease domains, and clinical guidelines may need updating to incorporate distributional targets alongside absolute thresholds for various conditions.
    \item \textbf{Health literacy and numeracy:} Patients with lower numeracy may find percentile‑based risk communication more intuitive than absolute probabilities across different diseases, but this requires careful design and validation.
    \item \textbf{Multiple biomarker tracking:} Simultaneously tracking sigma positions for different conditions (e.g., diabetes, hypertension, dyslipidemia) may create information overload for patients and clinicians.
\end{itemize}

\textbf{Future work} should include pilot studies to test patient comprehension, clinician usability, and clinical utility of sigma‑based feedback across different disease domains in preventive care settings.

\subsection{Novelty and differentiation}
Existing approaches use absolute risk estimates, static thresholds, or latent biological age constructs focused on specific diseases. Sigma-distribution differs by: (1) treating prevention as relative positioning within onset-time distributions applicable across multiple conditions, (2) making survival-conditioning explicit for various disease outcomes, (3) incorporating biomarker variability as first-class information across different risk indicators, (4) remaining agnostic to biological mechanism while applicable to various age-related diseases.

\section*{Acknowledgements}
None.

\section*{Funding}
This research did not receive any specific grant from funding agencies in the public, commercial, or not-for-profit sectors.

\section*{Declaration of competing interests}
The author declares no competing interests.

\section*{Declaration of generative AI and AI-assisted technologies}
During preparation of this work, the author used generative AI for language editing and formatting suggestions. After using this tool, the author reviewed and edited the content as needed and takes full responsibility for the content of the published article.

% \bibliographystyle{plain}
% \bibliography{references}

\begin{thebibliography}{99}

\bibitem{cox1972}
Cox DR. Regression models and life-tables.
\emph{Journal of the Royal Statistical Society: Series B}.
1972;34(2):187–220.

\bibitem{pencina2009}
Pencina MJ, D’Agostino RB, Larson MG, Massaro JM, Vasan RS.
Predicting the 30-year risk of cardiovascular disease.
\emph{Circulation}.
2009;119(24):3078–3084.

\bibitem{rothwell2010}
Rothwell PM.
Limitations of the usual blood-pressure hypothesis.
\emph{The Lancet}.
2010;375(9718):938–948.

\bibitem{monnier2006}
Monnier L, Mas E, Ginet C, et al.
Activation of oxidative stress by acute glucose fluctuations compared with sustained chronic hyperglycemia in patients with type 2 diabetes.
\emph{Diabetes Care}.
2006;29(2):455–460.

\bibitem{fries1980}
Fries JF.
Aging, natural death, and the compression of morbidity.
\emph{New England Journal of Medicine}.
1980;303(3):130–135.

\bibitem{fries2011}
Fries JF, Bruce B, Chakravarty E.
Compression of morbidity 1980–2011.
\emph{Journal of Aging Research}.
2011;2011:261702.

\bibitem{hillier2001}
Hillier TA, Pedula KL.
Characteristics of adults with newly diagnosed type 2 diabetes.
\emph{Diabetes Care}.
2001;24(9):1522–1527.

\bibitem{gregg2014}
Gregg EW, Zhuo X, Cheng YJ, Albright AL, Narayan KM.
Trends in lifetime risk and years of life lost due to diabetes in the USA, 1985–2011.
\emph{New England Journal of Medicine}.
2014;370(16):1515–1523.

\bibitem{seely2004}
Seely AJE, Macklem PT.
Complex systems and physiology.
\emph{Journal of Applied Physiology}.
2004;96(2):683–692.

\bibitem{levine2018}
Levine ME, Lu AT, Quach A, et al.
An epigenetic biomarker of aging for lifespan and healthspan.
\emph{Aging}.
2018;10(4):573–591.

\end{thebibliography}

\end{document}