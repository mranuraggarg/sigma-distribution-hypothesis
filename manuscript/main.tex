\documentclass[11pt]{article}
\usepackage{amsmath, amssymb}
\usepackage{geometry}
\usepackage{setspace}
\usepackage{hyperref}
\usepackage{graphicx}
\geometry{margin=1in}
\onehalfspacing

\title{The Sigma-Distribution Hypothesis: A Probabilistic Framework for Lifestyle Disease Onset, Aging, and Prevention}
\author{Anurag Garg}
\date{}

\begin{document}
\maketitle

\begin{abstract}
Lifestyle-related chronic diseases such as hypertension and type 2 diabetes follow probabilistic, age-dependent trajectories rather than deterministic paths. Current clinical practice emphasizes binary thresholds and absolute risk scores, but lacks a formal probabilistic framework for understanding how disease-free survival modifies future risk. We hypothesize that remaining disease-free with stable biomarkers shifts an individual's position rightward within the population distribution of disease onset times, a shift measurable in standard deviation ($\sigma$) units from the mean. This \emph{sigma-distribution hypothesis} integrates survival analysis, variance dynamics, and conditional probability to provide a mathematically coherent model for preventive success. The framework predicts that: (1) each disease-free year with stable biomarkers reduces the conditional hazard of near-term onset, (2) biomarker variability reduction amplifies this protective shift, and (3) sigma position serves as a communication tool for probabilistic prevention. This hypothesis offers testable predictions and could transform how prevention is quantified, communicated, and studied.
\end{abstract}

\section{Introduction}

The global burden of lifestyle-related chronic diseases—hypertension, type 2 diabetes, cardiovascular disease—continues to rise despite advances in understanding their pathophysiology \cite{rothwell2010, gregg2014}. Clinical prevention currently operates on two dominant paradigms: risk factor modification (lowering blood pressure, glucose, cholesterol) and absolute risk prediction (e.g., 10-year cardiovascular risk scores). While valuable, these approaches inadequately capture the \emph{probabilistic nature} of disease emergence over the lifespan.

Epidemiological models describe disease onset using hazard functions $\lambda(t)$, where risk accumulates with age but is modified by survival history \cite{cox1972, pencina2009}. Similarly, aging biology conceptualizes healthspan as a multidimensional trajectory with increasing variability over time \cite{levine2018}. Yet, no unified framework exists to translate these statistical and biological realities into a practical model for understanding preventive success.

We propose a novel synthesis: the \textbf{sigma-distribution hypothesis}. This framework posits that individuals occupy moving positions within population distributions of disease onset times, and that preventive success can be quantified as rightward movement in these distributions, measured in standard deviation units from the mean. This is not merely a statistical observation but a fundamental reconceptualization of how we understand disease delay and compression of morbidity \cite{fries1980}.

\section{The Hypothesis}

\subsection{Core Proposition}

The sigma-distribution hypothesis consists of three interconnected propositions:

\textbf{Proposition 1:} For lifestyle diseases with age-dependent onset, the population distribution of onset times can be locally approximated as Gaussian $T \sim \mathcal{N}(\mu, \sigma^2)$ within clinically relevant age windows (e.g., 40-70 years).

\textbf{Proposition 2:} An individual's position within this distribution, expressed as $z = (t_{\text{current}} - \mu)/\sigma$, dynamically updates with each disease-free year. Remaining disease-free while maintaining stable biomarkers causes $z$ to increase, representing a rightward shift toward later expected onset.

\textbf{Proposition 3:} Biomarker variability reduction (e.g., lower blood pressure variability, stable glucose) compresses the personal onset distribution, reducing left-tail probability mass and reinforcing rightward sigma shifts.

\subsection{Mathematical Formalization}

Let $T$ be the random variable representing age of disease onset for a specific condition in a reference population, with estimated parameters $\hat{\mu}$ (mean onset age) and $\hat{\sigma}$ (standard deviation). For an individual at current age $a$:

\[
z(a) = \frac{a - \hat{\mu}}{\hat{\sigma}} + \Delta z_{\text{cumulative}}
\]
where:
\begin{itemize}
    \item $a$ is current age
    \item $\hat{\mu}$ is estimated mean onset age
    \item $\hat{\sigma}$ is estimated standard deviation
    \item $\Delta z_{\text{cumulative}}$ accumulates from biomarker stability and risk factors
\end{itemize}
 
where $\Delta z_{\text{cumulative}}$ accumulates from:
\begin{enumerate}
\item Disease-free survival: $\Delta z_{\text{survival}} = f(T > a \mid \text{biomarkers})$
\item Biomarker stability: $\Delta z_{\text{stability}} = g(\text{Var}(\text{biomarkers}{[0,a]}))$
\item Risk factor levels: $\Delta z{\text{levels}} = h(\text{mean biomarker levels})$
\end{enumerate}

The conditional hazard at age $a$ becomes:

\[
\lambda(a \mid z) = \lambda_0(a) \cdot \exp(-\beta z)
\]
where:
\begin{itemize}
    \item $\lambda(a \mid z)$ is the conditional hazard at age $a$ given sigma position $z$
    \item $\lambda_0(a)$ is the baseline hazard at age $a$
    \item $\beta > 0$ represents the protective effect of rightward sigma positioning
    \item $\exp(-\beta z)$ modifies baseline hazard based on distribution position
\end{itemize}
where $\beta > 0$ represents the protective effect of rightward sigma positioning.

\section{Evaluation of the Hypothesis}

\subsection{Empirical Foundations}

The hypothesis integrates four established research domains:

\subsubsection{Survival Analysis and Conditional Probability}
Cox proportional hazards models inherently incorporate the concept of updating risk based on survival time \cite{cox1972}. Our framework makes this Bayesian updating explicit: $P(\text{early-onset} \mid T > t) < P(\text{early-onset})$.

\subsubsection{Variability as Risk Predictor}
Multiple studies demonstrate that biomarker variability predicts outcomes independent of mean levels:
\begin{itemize}
\item Blood pressure variability predicts stroke independent of mean BP \cite{rothwell2010}
\item Glycemic variability correlates with oxidative stress independent of HbA1c \cite{monnier2006}
\item Heart rate variability predicts mortality
\end{itemize}
These findings support Proposition 3: reducing variability should compress onset distributions.

\subsubsection{Compression of Morbidity}
Fries' compression of morbidity hypothesis \cite{fries1980, fries2011} aligns perfectly with sigma shifts: delayed onset without extended lifespan compresses illness into later years. Our framework quantifies this delay in distributional terms.

\subsubsection{Early vs. Late-Onset Phenotypes}
Clinical evidence shows early-onset diseases represent more aggressive phenotypes \cite{hillier2001}. Remaining disease-free reduces the posterior probability of belonging to this subgroup, equivalent to rightward sigma movement.

\subsection{Theoretical Consistency}

The hypothesis is consistent with:
\begin{itemize}
\item \textbf{Statistical theory}: Gaussian approximations are valid locally even for non-normal processes
\item \textbf{Systems physiology}: Stable control loops dissipate less energy and avoid runaway states \cite{seely2004}
\item \textbf{Aging biology}: Biological age estimators use multivariate distributions \cite{levine2018}
\item \textbf{Risk communication}: Relative positioning is more intuitive than absolute probabilities
\end{itemize}

\section{Illustrative Proof-of-Concept Using a Single Biomarker}

To demonstrate the computational feasibility of the sigma-distribution framework without making clinical efficacy claims, we present an illustrative proof-of-concept using a single biomarker: glycated hemoglobin (HbA1c). HbA1c is selected solely because it is widely measured, biologically stable over short timescales, age-dependent, and available in large public epidemiologic datasets \cite{NHANESref}. This demonstration is intended to show operationalization of sigma positioning, not to propose HbA1c as a sufficient or superior predictor of disease onset.

\subsection{Reference Distribution Estimation}

Using large publicly available population datasets such as the National Health and Nutrition Examination Survey (NHANES) \cite{NHANESref}, age-stratified reference distributions of HbA1c can be estimated among individuals without diagnosed diabetes. For each age bin (e.g., 5-year intervals from 40-75 years) or via smoothed age-dependent modeling, the mean $\mu_{\text{HbA1c}}(a)$ and standard deviation $\sigma_{\text{HbA1c}}(a)$ are computed from the non-diabetic population subset. This approach deliberately avoids assuming global normality of HbA1c values across the lifespan. Instead, Gaussian approximations are applied locally within clinically relevant age ranges, consistent with standard epidemiologic practice.

\subsection{Sigma Position Computation}

For an individual of age $a$ with HbA1c value $x$, the sigma position is defined as:
\[
z_{\text{HbA1c}}(a) = \frac{x - \mu_{\text{HbA1c}}(a)}{\sigma_{\text{HbA1c}}(a)}
\]

This quantity represents the individual's standardized position within the age-matched reference distribution. Importantly, the same absolute HbA1c value may correspond to different sigma positions at different ages due to shifts in $\mu(a)$ and $\sigma(a)$.

\subsubsection{Computational Example}

Consider a 55-year-old individual with HbA1c = 5.8\%. From illustrative NHANES-derived values (not actual analysis, for demonstration only), for age 55, $\mu = 5.6\%$ and $\sigma = 0.35\%$. Their sigma position is:
\[
z = \frac{5.8 - 5.6}{0.35} = +0.57
\]
This places them approximately at the 72nd percentile of their age-matched reference distribution. If they maintain this position until age 60 while remaining diabetes-free, their cumulative sigma shift $\Delta z$ increases, reflecting reduced conditional hazard.

\begin{table}[h]
\centering
\caption{Illustrative HbA1c Sigma Positioning at Different Ages}
\begin{tabular}{c c c c c}
\hline
Age (years) & HbA1c (\%) & $\mu(a)$ (\%) & $\sigma(a)$ (\%) & $z$ \\
\hline
40 & 5.6 & 5.4 & 0.3 & +0.67 \\
60 & 5.6 & 5.7 & 0.4 & -0.25 \\
40 & 5.9 & 5.4 & 0.3 & +1.67 \\
60 & 5.9 & 5.7 & 0.4 & +0.50 \\
\hline
\end{tabular}
\label{tab:hba1c_example}
\end{table}

Table \ref{tab:hba1c_example} illustrates how identical absolute HbA1c values correspond to different sigma positions at different ages, highlighting the age-conditioned interpretation central to the sigma-distribution hypothesis.

\subsection{Conceptual Implications}

This simple computation illustrates four key features of the sigma-distribution hypothesis:

\begin{enumerate}
\item \textbf{Age-conditioned interpretation}: Identical biomarker values acquire different probabilistic meanings depending on age, reinforcing the inadequacy of fixed thresholds across the lifespan.

\item \textbf{Implicit survival conditioning}: Remaining disease-free at age $a$ means the individual's onset time $T$ satisfies $T > a$. This observation updates the posterior distribution $P(T \mid T > a)$, effectively truncating the left tail and shifting the conditional mean rightward. This statistical fact underlies the intuitive notion that "each disease-free year is protective."

\item \textbf{Extensibility}: While demonstrated here for a single marker, the same framework naturally extends to multivariate sigma positioning using correlated biomarker sets. A composite $z$-score could be computed as a weighted sum of individual biomarker positions.

\item \textbf{Variance as information}: The $\sigma(a)$ parameter itself contains predictive information about population heterogeneity. Populations with wider biomarker distributions may have more heterogeneous biological aging trajectories, and individuals with stable biomarkers (low intra-individual variance) may experience slower sigma drift.
\end{enumerate}

No outcome prediction, hazard modeling, or intervention effect is claimed or tested in this illustrative analysis. The sole purpose is to demonstrate that sigma positioning can be computed transparently using real-world data, thereby grounding the theoretical framework in operational reality.

\subsection{Scope and Limitations of the Demonstration}

This proof-of-concept is intentionally limited in scope. HbA1c alone does not capture the full complexity of metabolic risk, and the analysis does not address competing risks, longitudinal variability, or multimarker interactions. Furthermore, the Gaussian approximation, while reasonable for local age windows, may not hold for all biomarkers or extreme values. These extensions are the subject of future work. The present demonstration serves only to establish feasibility and interpretability of sigma-based positioning within an epidemiologic context.

\subsection*{Figure 1 Concept Description}

\textbf{Figure 1. Age-conditioned sigma positioning of HbA1c.} Schematic representation showing age-specific mean HbA1c values (solid line) with corresponding $\pm1\sigma$ bands (shaded region) estimated from a public epidemiologic dataset. Individual measurements (points) are shown with their sigma positions relative to age-matched distributions. The figure demonstrates that identical absolute HbA1c values correspond to different sigma positions at different ages, highlighting the probabilistic, age-dependent interpretation central to the sigma-distribution hypothesis. This figure is intended for conceptual illustration only and does not represent outcome prediction or clinical validation.
\begin{figure}
    \centering
    \includegraphics[width=\linewidth]{figure1_sigma_hba1c_sample.png}
    \caption{\textbf{Figure 1. Schematic illustration of sigma positioning.} (Top) Age-specific reference distributions of HbA1c showing increasing mean and variance with age. The $\pm1\sigma$ band represents the population distribution width. (Middle) Two individuals with identical HbA1c values (5.8\%) at different ages occupy different sigma positions: z = +1.45 at age 55 vs. z = +0.73 at age 75. (Bottom) Sigma trajectory over time: maintaining stable biomarker levels relative to age-matched peers results in rightward sigma drift (increasing z-scores), representing reduced conditional hazard. All values are illustrative for conceptual demonstration.}
    \label{fig:placeholder}
\end{figure}
\section{Predictions and Testable Consequences}

\subsection{Immediate Testable Predictions}

\begin{enumerate}
\item \textbf{Prediction 1}: In longitudinal cohorts, individuals who maintain $z > +1$ at age 60 should have $\geq 50\%$ lower 10-year incidence than those with $z < 0$, controlling for traditional risk factors.

\item \textbf{Prediction 2}: Biomarker variability measures (e.g., visit-to-visit BP variability, glucose SD) will correlate more strongly with sigma position than with mean biomarker levels.

\item \textbf{Prediction 3}: Interventions that reduce biomarker variability should produce greater rightward sigma shifts than interventions that only lower mean levels.

\item \textbf{Prediction 4}: Sigma position at midlife (e.g., age 50) will correlate with biological age estimates (PhenoAge, DunedinPACE) independently of chronological age.

\end{enumerate}

\subsection{Clinical and Research Implications}

\begin{enumerate}
\item \textbf{New trial endpoints}: Instead of just hazard ratios, trials could report mean sigma shifts in intervention vs. control groups.

\item \textbf{Personalized feedback}: Individuals could track their sigma position over time as a motivational tool.

\item \textbf{Risk communication}: "You've moved from the 50th to 70th percentile of onset distribution" may be more meaningful than "Your 10-year risk is 15%."

\item \textbf{Precision prevention}: Different interventions could be recommended based on whether someone needs to reduce variability vs. shift means.
\end{enumerate}

\section{Discussion}

\subsection{Novelty and Differentiation}

The sigma-distribution hypothesis differs from existing models in several ways:

\begin{itemize}
\item \textbf{Vs. absolute risk scores}: Focuses on relative position within distributions rather than absolute probabilities
\item \textbf{Vs. biological age clocks}: Agnostic to mechanisms, focused on functional outcomes
\item \textbf{Vs. traditional survival analysis}: Makes conditional probability updates explicit and visualizable
\item \textbf{Vs. variability research}: Integrates variability into a comprehensive positioning framework
\end{itemize}

\subsection{Potential Limitations}

\begin{enumerate}
\item \textbf{Distribution assumptions}: Not all onset distributions are Gaussian, though local approximations are reasonable.
\item \textbf{Competing risks}: Death from other causes might preclude observing disease onset.
\item \textbf{Measurement challenges}: Requires longitudinal biomarker data to calculate variability.
\item \textbf{Population specificity}: $\mu$ and $\sigma$ may vary across ethnicities, cohorts, and eras.
\end{enumerate}

\subsection{Future Research Directions}

\begin{enumerate}
\item Develop algorithms to estimate sigma positions from existing cohort data
\item Conduct randomized trials using sigma shifts as endpoints
\item Create visualization tools for clinicians and patients
\item Explore genetic and molecular correlates of sigma position
\item Extend to multimorbidity (multiple disease distributions simultaneously)
\end{enumerate}

\section{Conclusion}

The sigma-distribution hypothesis provides a mathematically rigorous yet intuitively accessible framework for understanding lifestyle disease prevention. By conceptualizing disease-free survival as rightward movement within onset distributions, it bridges epidemiological theory, clinical practice, and patient understanding. The hypothesis makes testable predictions and, if validated, could transform how we quantify, communicate, and achieve preventive success. Rather than chasing perfect biomarkers or zero risk, the framework emphasizes continuous rightward movement—a probabilistic journey away from early onset toward compressed morbidity.

\begin{thebibliography}{99}

\bibitem{cox1972}
Cox DR. Regression models and life-tables. \emph{Journal of the Royal Statistical Society: Series B}. 1972.

\bibitem{pencina2009}
Pencina MJ et al. Predicting the 30-year risk of cardiovascular disease. \emph{Circulation}. 2009.

\bibitem{rothwell2010}
Rothwell PM. Limitations of the usual blood-pressure hypothesis. \emph{Lancet}. 2010.

\bibitem{monnier2006}
Monnier L et al. Glycemic variability and oxidative stress. \emph{Diabetes Care}. 2006.

\bibitem{fries1980}
Fries JF. Aging, natural death, and the compression of morbidity. \emph{New England Journal of Medicine}. 1980.

\bibitem{fries2011}
Fries JF et al. Compression of morbidity 1980–2011. \emph{Journal of Aging Research}. 2011.

\bibitem{hillier2001}
Hillier TA, Pedula KL. Characteristics of adults with newly diagnosed type 2 diabetes. \emph{Diabetes Care}. 2001.

\bibitem{gregg2014}
Gregg EW et al. Trends in lifetime risk and years of life lost due to diabetes. \emph{New England Journal of Medicine}. 2014.

\bibitem{seely2004}
Seely AJE, Macklem PT. Complex systems and physiology. \emph{Journal of Applied Physiology}. 2004.

\bibitem{levine2018}
Levine ME et al. An epigenetic biomarker of aging for lifespan and healthspan. \emph{Aging}. 2018.

\bibitem{NHANESref}
National Health and Nutrition Examination Survey. Centers for Disease Control and Prevention. Available: https://www.cdc.gov/nchs/nhanes/

\end{thebibliography}

\end{document}