\documentclass[11pt]{article}

\usepackage{graphicx}
\usepackage{geometry}
\usepackage{setspace}
\geometry{margin=1in}
\onehalfspacing

\begin{document}

\section*{Supplementary Material: Illustrative Computational Figures}

The following figures are provided solely for conceptual illustration of the proposed sigma-distribution framework. They do not constitute clinical validation, hypothesis testing, or predictive evaluation, and are included to aid interpretability of the mathematical and population-level ideas discussed in the main text.

The figures are organized to mirror the logical structure of the hypothesis. 
The first two figures illustrate why absolute biomarker thresholds are insufficient across the lifespan and motivate age-conditioned positioning. 
The third figure provides a controlled synthetic illustration showing that re-parameterization into sigma space is internally coherent under a known data-generating process. 
Together, these figures support the conceptual plausibility of sigma-based positioning without asserting empirical validation.

\begin{figure}[htbp]
\centering
\includegraphics[width=0.9\textwidth]{figure_nhanes_biomarker_vs_age.png}
\caption{\textbf{Illustrative population cross-section: biomarker level versus age.}
A cross-sectional scatter plot with a nonparametric smoother shows that population biomarker distributions shift with age while retaining substantial dispersion at each age. This motivates age-conditioning when expressing an individual’s relative position within a population distribution. No incidence, progression, or onset-time inference is implied.}
\label{fig:supp_nhanes_age}
\end{figure}

\begin{figure}[htbp]
\centering
\includegraphics[width=0.9\textwidth]{figure_sigma_histogram.png}
\caption{\textbf{Illustrative distribution of sigma positions.}
Histogram of age-conditioned sigma (z-score) values demonstrates that sigma positioning yields a numerically stable and interpretable coordinate system across the population, with symmetric tails after conditioning. This figure illustrates feasibility and interpretability of the transformation rather than disease risk prediction.}
\label{fig:supp_sigma_hist}
\end{figure}

\begin{figure}[htbp]
\centering
\includegraphics[width=0.9\textwidth]{figure_synthetic_roc.png}
\caption{\textbf{Synthetic illustration under a known generative model.}
ROC curves compare representations based on an absolute biomarker scale versus an age-conditioned sigma coordinate in a controlled synthetic dataset where the data-generating process is specified. The purpose is to demonstrate internal coherence of the re-parameterization and illustrate how representation alone can influence discrimination in simulation. This is not external validation and does not imply clinical utility.}
\label{fig:supp_synthetic_roc}
\end{figure}

\end{document}